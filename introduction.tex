\subsection{About ARFC}
\begin{frame}
  \frametitle{Advanced Reactors and Fuel Cycles group (PI: Kathryn Huff)}
               \begin{figure}[t]
                \vspace*{-0.1in}
                \includegraphics[height=0.71\textwidth]{./images/arfc1.png}
               \end{figure}            
\end{frame}

\begin{frame}
  \frametitle{Insights at Disparate Scales}
               \begin{figure}[t]
                \vspace*{-0.1in}
			\hspace*{-0.35in}
                \includegraphics[height=0.5\textwidth]{./images/synergy.png}
               \end{figure}            
\end{frame}

\subsection{Fission basics}
\begin{frame}
  \frametitle{Nuclear Fission Reaction}
               \begin{figure}[t]
                \vspace*{-0.1in}
			\hspace*{-0.35in}
                \includegraphics[height=0.25\textwidth]{./images/800px-Fission.png}
               \end{figure}            
\end{frame}

\begin{frame}
  \frametitle{Nuclear Fission Chain Reaction}
               \begin{figure}[t]
                \vspace*{-0.1in}
			\hspace*{-0.35in}
                \includegraphics[height=0.7\textwidth]{./images/715px-Chainreaction.png}
               \end{figure}            
\end{frame}

\begin{frame}
  \frametitle{Nuclear Power Plant}
  \animategraphics[width=\textwidth,autoplay,loop]{6}{./images/gif/frame-}{0}{23}
\end{frame}

\subsection{Motivation}
\begin{frame}
  \frametitle{Why Molten Salt Reactors?}
                  \vspace*{-0.1in}
              \begin{block}{Main advantages of liquid-fueled \glspl{MSR} \cite{elsheikh_safety_2013}}
               \begin{enumerate}
                \item High coolant temperature (600-750$^{\circ}$C).
                \item Various fuels can be used ($^{235}$U, $^{233}$U, Thorium, U/Pu).
                \item Increased inherent safety.
                \item High fuel utilization $\Rightarrow$ less nuclear waste generated.
                \item Online reprocessing and refueling.
               \end{enumerate}
               \end{block}
                  \vspace*{-0.1in}               
               \begin{block}{Main advantages of \gls{MSBR} \cite{robertson_conceptual_1971}}
               \begin{enumerate}
                \item Produces more fissile material than it consumes (breeding ratio 1.06).
                \item Thorium cycle limits plutonium and minor actinides.
                \item Could transmute spent fuel from existing \gls{NPP}.
               \end{enumerate}
               \end{block}

\end{frame}

\begin{frame}
  \frametitle{Challenges in simulation \gls{MSR}}
                  \vspace*{-0.05in}
               \begin{enumerate}
                \item Contemporary burnup codes cannot treat fuel movement.
                \item Neutron precursor location is hard to estimate.
                \item Operational and safety parameters change during reactor operation.
                \item Power generation strongly depends on fuel temperature and vica versa.
               \end{enumerate}

           \begin{figure}[t]
                \vspace*{-0.3in}
			\hspace*{-0.2in}
                \includegraphics[height=0.5\textwidth]{./images/coupled_physics.png}
		\vspace*{-0.05in}
		\caption{Multiphysics simulation scheme for \gls{MSR} (Courtesy of Manuele Aufiero,2012).}
     	 \end{figure}               
\end{frame}

\begin{frame}
  \frametitle{Research objectives}
                  \vspace*{-0.1in}
              \begin{block}{Goal \#1: Tool for online reprocessing depletion simulation (SaltProc)\cite{rykhlevskii_saltproc}}
               \begin{enumerate}
                \item Create high-fidelity full-core 3-D model of MSBR without any approximations using the continuous-energy SERPENT 2 Monte Carlo physics software \cite{leppanen_serpent_2012}.
                \item Develop online reprocessing simulation code, SaltProc, which expands the capability of SERPENT for simulation liquid-fueled \gls{MSR} operation.
                \item Analyse \gls{MSBR} neutronics and fuel cycle to find the equilibrium core composition and core depletion.
                \item Compare predicted operational and safety parameters of the \gls{MSBR} at both the initial and equilibrium states.
               \end{enumerate}
               \end{block}

              \begin{block}{Goal \#2: Tool for multiphysics simulation of \gls{MSR} (Moltres)\cite{lindsay_introduction_2018}}
               \begin{enumerate}
                \item Demonstrate steady-state coupling of neutron fluxes, precursors, and temperature for thermal \gls{MSR} design.
                \item Implement advective movement of delayed neutron precursors.
                \item Demonstrate capabilities with 2D axisymmetric and 3D structured/unstructured mesh.
               \end{enumerate}
               \end{block}


              
\end{frame}
